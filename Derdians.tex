%% start of file 'cv.tex'.
\documentclass[11pt,a4paper]{article}

% character encoding
\usepackage[utf8]{inputenc}                   % replace by the encoding you are using

%----------------------------------------------------------------------------------
%            content
%----------------------------------------------------------------------------------
\begin{document}

\section{Les Derdians}

\subsection{Situation}

Vous vivez dans un pays appelé Derdia.
Le village dans lequel vous habitez est séparé de la ville la plus proche où se trouve le marché par une profonde vallée.
Aller au marché signifie deux jours de marche. Si vous disposiez d'un pont au-dessus de la vallée, cinq heures suffiraient.

Votre gouvernement a signé un contrat avec une firme étrangère afin qu'elle vienne vous apprendre la construction de ponts.
Les habitants de votre village seront alors les premiers ingénieurs de Derdia.
Après avoir construit ce premier pont avec l'aide d'experts étrangers, vous allez pouvoir en construire d'autres dans tout le pays afin de faciliter la vie de vos concitoyens.

Le pont sera construit en utilisant du carton, des crayons, des règles et des ciseaux.
Vous connaissez le matériel et les outils, mais pas les techniques de construction

\subsection{Règles sociales}

Les Derdians ont pour habitude de se toucher mutuellement.
Leur communication repose sur le contact physique.
Ne pas toucher quelqu'un à qui l'on parle est très grossier.
Si cela se produit, la personne insultée se met à crier très fort.

Lorsque vous vous joignez à un groupe, il vous suffit de vous accrocher à l'un des membres pour être instantanément inclus dans la conversation.
Saluer les personnes que vous rencontrez est essentiel, même si vous ne faites que les croiser.

Le salut derdian consiste en un contact d'épaule.
Les deux personnes doivent mettre leurs épaules droites en contact.
Toute autre forme d'embrassade est une insulte! Serrer la main, par exemple, est l'une des insultes les plus graves dans ce pays.
Si un Derdian est insulté parce qu'il n'a pas été salué comme il convient ou touché pendant qu'on lui parlait, il se met à crier très fort.

Un homme de Derdia n'entrera jamais en contact avec un autre homme à moins qu'il ne lui soit présenté par une femme, qu'elle soit de Derdia ou pas.

Les Derdians n'emploient pas le terme \emph{non}. Ils disent toujours \emph{oui} mais, lorsqu'ils veulent dire \emph{non}, ils accompagnent leur \emph{oui} de hochements de tête négatifs.

Dans leur travail, les Derdians touchent aussi beaucoup.
Certains outils sont propres au sexe masculin, d'autres au sexe féminin.
Les objets féminins (une règle par exemple) ne peut être correctement utilisés que par une femme; Les objets masculins (un stylo par exemple) ne peut être correctement utilisé que par un homme.
Tout le monde peut manipuler n'importe quel objet, mais les hommes utiliseront de manière \emph{stupide} les objets féminins (et inversement).

Le peuple Derdians est un peuple solidaire. Si un Derdians est insulté, les autres derdians présents dans la salle crient également.
 
 \subsection{Comportement avec les étrangers}
 
 Les Derdians apprécient la compagnie.
 Par conséquent, ils aiment les étrangers.
 Mais ils sont également très fiers d'eux-mêmes et de leur culture.
 Ils savent qu'ils ne seront jamais capables de construire un pont sans aide.
 Ce n'est pas pour autant qu'ils jugent supérieures la culture et l'éducation des étrangers; pour eux, la construction de ponts est tout simplement un art qu'ils ne maîtrisent pas.
 Ils attendent des étrangers qu'ils s'adaptent à leur culture.
 Or, dans la mesure où leur comportement leur paraît tout à fait naturel, ils sont incapables de l'expliquer aux experts (ce point est TRES important).

 \emph{Les Derdians communiquent plus par le toucher que par la parole. Ils sont peu bavards. Ne pas toucher quelqu'un lorsqu'on lui adresse la parole ou ne pas saluer une personne que l'on croise est très grossier et énerve les Derdians au plus haut point (ils se mettent à crier).}

\pagebreak
 
\section{Les ingénieurs}

\subsection{Situation}

Vous êtes une équipe internationale d'ingénieurs travaillant pour une entreprise de construction multinationale.
Votre entreprise vient de signer un contrat important avec le gouvernement de Derdia, par lequel elle s'engage à apprendre aux Derdians la construction de ponts.
Le contrat stipule que vous devez impérativement respecter les délais convenus, sinon le contrat sera rompu et vous vous retrouverez au chômage.

Derdia est un pays montagneux, jalonné de canyons et de profondes vallées, mais privé de pont.
Par conséquent, il faut plusieurs jours aux Derdians pour se rendre de leurs villages au marché de la ville la plus proche. Grâce à un pont, on estime que le trajet pourrait être fait en moins de cinq heures.

Etant donné le nombre de canyons et de rivières dans le pays, vous ne pouvez vous contenter de construire un pont puis de repartir.
Vous allez devoir apprendre aux Derdians les techniques de construction.

\subsection{Construction du pont}

Le pont sera symbolisé au moyen d'une construction en carton entre deux chaises ou deux tables séparées par une distance d'environ 80 cm.
Il devra être stable. Une fois terminé, il devra pouvoir supporter le poids des outils ayant servi à sa construction.

Il ne suffira pas de découper les pièces du pont puis de les assembler dans le village, car cela ne permettrait pas aux Derdians d'apprendre les techniques de construction.
Ces derniers devront pouvoir assister à toutes les phases de la construction.

Chacun des éléments du pont devra être dessiné au crayon et à la règle avant d'être découpé à l'aide des ciseaux.

\end{document}


